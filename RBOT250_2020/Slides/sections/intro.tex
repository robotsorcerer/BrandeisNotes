\section{Introduction}

\setbeamertemplate{blocks}[rounded][shadow=true]
\subsection{Textbooks}
\begin{frame}
	\frametitle{Standard Texts -- Modeling and Control}
	\begin{block}{Robot Modeling and Control}
		Spong, Mark W., Seth Hutchinson, and Mathukumalli Vidyasagar. Robot modeling and control. Vol. 3. New York: Wiley, 2006.
	\end{block}
	\begin{block}{Mathematical Modeling of Robots}
		Murray, R. M., Li, Z., \& Sastry, S. S. (1994). A Mathematical Introduction to Robotic Manipulation. In Book (Vol. 29). https://doi.org/10.1.1.169.3957
	\end{block}
\end{frame}

\begin{frame}
	\frametitle{Texts -- Modeling, Control, and Mechanisms}
	\begin{block}{Robot Modeling and Control}
		Lynch, K. M., \& Park, F. C. (2017). Modern Robotics Mechanics, Planning, and Control.
	\end{block}
	%
	\begin{tcolorbox}[coltitle=blue!50!black,colframe=blue!25,title=Mechanisms' Kinematic Geometry]
		Hunt, Kenneth H., and Kenneth Henderson Hunt. Kinematic geometry of mechanisms. Vol. 7. Oxford University Press, USA, 1978.
	\end{tcolorbox}
\end{frame}


\begin{frame}
	\frametitle{Texts -- Screws and Kinematics}
	\begin{tcolorbox}[coltitle=blue!50!black,colframe=blue!25,title=Screw Theory]
		Ball, Robert Stawell. A Treatise on the Theory of Screws. Cambridge university press, 1998.
	\end{tcolorbox}
	%
	\begin{tcolorbox}[arc=4mm, 		coltitle=blue!50!black,colframe=blue!25,title=Mechanisms' Kinematic Geometry]
		%
		Hunt, K. H. (2019). Structural Kinematics of In- Parallel-Actuated Robot-Arms. 105(December 1983), 705–712.
	\end{tcolorbox}
\end{frame}
