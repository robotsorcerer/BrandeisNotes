\section{Future Work}

\begin{frame}
	\begin{itemize}
		\item We have presented an optimal and an adaptive 
		vision-based control approach in IMRT 
		%for achieving better positioning accuracy during non-invasive intensity modulated radiotherapy. 
		\item Soft robot mechanism was designed to help achieve necrosis in malignant tumor  cells  in the cranial region. 
		%
		\item Given nonlinearity of these semi-rigid and continuum bodies, we will
		
		\begin{itemize}
			\item conduct multiDOF simulations and verification of better modeling approaches 
			%		
		\end{itemize}
	\end{itemize}
\end{frame}

\begin{frame}
	\frametitle{Future Work}
	\begin{itemize}
			\item Adopt best practices from one of the SOTA modeling approaches viz
			
			\vspace{0.3cm}
			\begin{itemize}
				\item 
				\textit{finite element methods}~\cite{Coevoet2017FEM,JamesBern}, 
				\vspace{0.3cm}
				\item constant curvature approach~\cite{godage2016dynamics}, 
				\vspace{0.3cm}
				\item the continuous Cosserat approach~\cite{Renda2014Cosserat}, and 
				%
				\vspace{0.3cm}
				\item the multi-body hyper-redundant model~\cite{kang12hyper}. 
			\end{itemize}
	\end{itemize}
\end{frame}

There has been recent work that unifies the discrete Cosserat approach~\cite{renda16discrete} with traditional rigid robotics~\cite{Renda2018ICRA}. We hope to build on this generalization of Brockett's theory of robotics in building multi-DOF joints as well as adopt the ``geometric exact beam" model originally proposed by Simo and Vu-LOCQ ~\cite{simo88}.


\begin{frame}
\frametitle{References}
\centering

\end{frame}