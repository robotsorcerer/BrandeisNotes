
\subsection{Prerequisites}
\begin{frame}
	\frametitle{Prerequisites}
	 	\begin{itemize}
	 		 \item A Working Knowlegde of C++: Occasionally, we would dabble into using the C++ 1z standards in our coding styles
	 		%
	 		\vspace{0.5em}
	 		%
	 		 \item A working knowledge of the python programming language
	 		%
	 		\vspace{0.5em}
	 		%
	 		 \item A working knowledge of the robot operating system (ROS) middleware.
	 		%
	 		\vspace{0.5em}
	 		%
	 		 \item To ease setup for labs, a dockerized environment has been provided for you that has all the tools you need to get a jumpstart most of the lab exercises in the notes.
	 	\end{itemize}
\end{frame}

\begin{frame}
	\frametitle{Loading the Docker Environment}
	\begin{itemize}
		\item Ensure you have a Ubuntu OS. For now, any distro from 14.04+ would do.
		%
		\vspace{.5em}
		%
		\item To download and install the Ubuntu OS, hop over to the Ubuntu download page: \href{https://ubuntu.com/download/desktop}{https://ubuntu.com/download/desktop} and follow the download and installation instructions 
		% \href
		\vspace{.5em}
		\item When you are done installing ubuntu, be sure to install the docker environment 
		%
		\vspace{.5em}
		%
		\item Go to \href{https://download.docker.com/linux/ubuntu/dists/}{ \textsf{this webpage}}, choose your Ubuntu version, browse to  \textsf{pool/stable/, choose amd64, armhf, arm64, ppc64el, or s390x}, and download the  \textsf{.deb} file for the Docker Engine - Community version you want to install.
	\end{itemize}
\end{frame}

\begin{frame}
	\begin{itemize}
		\item Install Docker Engine - Community, changing the path below to the path where you downloaded the Docker package.
		%
		\vspace{.5em}
		%
		\begin{itemize}
			\item \textsf{sudo dpkg -i /path/to/package.deb}
		\end{itemize}
	%
		\item Confirm that your installation runs by testing the \textsf{hello-world-run} image:
			\textsf{docker run hello-world}
			%
			\vspace{.5em}
			%
		\item Further instructions can be found on this \href{https://docs.docker.com/install/linux/docker-ce/ubuntu/}{\bl{webpage}}.
	\end{itemize}
\end{frame}

\begin{frame}
\begin{itemize}
	\item When you are done, there is a docker image that is already prepared for your use for most of the simulations we would use in this course.
	%
	\vspace{.5em}
	%
	\item It can be pulled like so:
	%
	\vspace{.5em}
	%
	\begin{itemize}
		\small \item \textsf{``docker pull lakehanne/brandeis:melodic"}
	\end{itemize}
	%
	\vspace{.5em}
	%
	\item Run the image:
	%
	\vspace{.5em}
	%
	\small \textsf{``docker run -ti --rm lakehanne/brandeis:melodic -v /tmp/.X11-unix:/tmp/.X11-unix:ro -e DISPLAY=\$DISPLAY --privileged -v /dev/bus/usb:/dev/bus/usb"}
	%
	\vspace{.5em}
	%
	\item This would launch the image together with usb access and access to your xorg server.  The ros installation is at \textsf{``/opt/ros/melodic"} and the catkin workspace is located at \textsf{``/home/rbot250/catkin\_ws/src"}. This is the directory from which all tutorials shall be launched.
\end{itemize}
\end{frame}

\begin{frame}
\frametitle{ROS Introduction}
	\begin{itemize}
		\item An easier way to run would be to launch the `docker-run` executable available here: `https://github.com/lakehanne/Shells/blob/master/docker-run'.
		%
		\item Follow the instructions that the bash script gives you 
		%
		\item Note that to compile, I have installed the catkin-build tools globally in the image which you can use as follows:
		\begin{itemize}
			\item `catkin build'
			%
			\item You can also run catkin build with the alias `cb'
			%
			\item To compile just a single package, say dr\_kdl, 		run `cb dr\_kdl'
		\end{itemize}
		
		\item Now that you have the ros environment setup, why don't you start playing around with the tutorials at \textsf{\href{http://wiki.ros.org/ROS/Tutorials}{\bl{ROS Tutorials Page}}}.
	\end{itemize}
\end{frame}
