\documentclass{article}
\usepackage{hyperref}
\usepackage{titling}

\title{RBOT 101 - Foundations: Introduction}
\posttitle{\par\end{center}}
  
\predate{}
\date{}
\postdate{}

\preauthor{}
\author{}
\postauthor{}

\begin{document}
\maketitle

\section{Introduction}
Welcome to the Foundations course for robotics. The purpose of this course is to refresh your memory on mathematical concepts to which you should already have encountered in past courses or avenues of study. As a start, you will find an assessment pre-test that you will complete now, and then again at the end of the course. The purpose of the survey is to allow me to understand where I need to focus attention. By the end of the course, you should have a workable knowledge of multiple mathematical areas that underly much of robotics.

Throughout this course, I assume that you have some level of understanding for each of the mathematical areas covered. In each area, I will provide additional references that will allow you to explore those areas in depth. As a graduate student, the expectation of you is that you will have the desire and capability of chasing down areas of knowledge that are necessary but in which you may have very little expertise. This course is intended to help you in the areas of mathematics that are important for robotics.

\section{Using Colab}
To facilitate this course, I'll be using Google Colab, which you can access at \href{http://colab.research.google.com}{Google Colab}. You login using a Google account, which you can create for free if you don't have one. Colab is an important tool in Machine Learning and other data-oriented topics and will be useful for us in discussing mathematical topics. Colab is based on python and works on top of Jupyter or IPython notebooks. The point of using Colab is to allow you to create responses to assignments that you can then send, as iPython notebooks, to me for evaluation.

Take some time to create a Colab notebook and play around with the interface. The beauty of Colab is that you can embed text and python code together. The text can be marked up, to a limited capability. The ability to interleave text and code makes for a wonderful interface for explaining code with actual text. While not Colab related, a great source for this concept can be found at \href{http://literateprogramming.com/}{Literate Programming} which is based on Donald Knuth's notion of creating readable documentation associated with the code created in the order in which the programmer creates the code (not necessarily linear).

So let's get started! We'll start with Calculus. This seems a strange place to start but it becomes very important in robotics. We'll review calculus overall and then focus on differentiation and integration.

\end{document}
