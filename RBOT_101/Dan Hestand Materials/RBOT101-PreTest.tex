\documentclass{article}
\usepackage{amsmath}
\usepackage{mathtools}
\usepackage{tikz}
\usepackage{enumerate}

\title{RBOT 101 - Foundations Mathematics Pre-test}

\begin{document}
\maketitle
The following test is meant to test your knowledge in key areas of mathematics useful in robotics. If you are unfamiliar with or don't know how to solve a particular problem, just write ``unknown'' for the answer. Some of this material may be familiar to some students and completely unknown to others: the intent is to provide the instructor with sufficient understanding of what is unfamiliar to ensure adequate coverage on those topics.
\section{Linear Algebra}
\begin{enumerate}[\thesection .1]
\item Perform the following multiplication. If you cannot, then explain why.
  \begin{equation*}
    \begin{bmatrix*}[r]
       1 & 2 & 0 \\
       10 & -1 & -10 \\
       -1 & 3 & 11
    \end{bmatrix*}
    \begin{bmatrix*}[r]
      2 & 3 \\
      1 & 0
    \end{bmatrix*}
  \end{equation*}

\item Perform the following multiplication. If you cannot, then explain why.
  \begin{equation*}
    \begin{bmatrix*}[r]
      1 & 2 & 9 \\
      3 & 1 & 0 \\
    \end{bmatrix*}
    \begin{bmatrix*}[r]
      1 \\
      3 \\
      5
    \end{bmatrix*}
  \end{equation*}

\item Given $A = \big(\begin{smallmatrix*}[r] a & b & c\\ d & e & f\\ g & h & k \end{smallmatrix*}\big)$, compute
  $det(A)$

\item Given $A$ from the previous problem, compute $tr(A)$.

\item Given $A$ as before but with all off-diagonal elements equal to $0$, compute $A^{-1}$.

\item Solve the given system of equations. If no solution is possible, explain why. If more than one solution is available, provide all solutions.
    \begin{alignat}{7}
        x_1 \quad &+ \quad &x_2 \quad &+ \quad &8 x_3 \quad &= \quad 0 \\
        2 x_1 \quad &+ \quad &3 x_2 \quad &+ \quad &19 x_3 \quad &= \quad 0 \\
        x_1 \quad &+ \quad &2 x_2 \quad &+ \quad &11 x_3 \quad &= \quad 0
    \end{alignat}

  \item Solve the given system of equations, if possible. If more than one solution is available, provide all solutions.
    \begin{alignat}{5}
      5 x_1 \quad &+ \quad &14 x_2 \quad &= \quad -35 \\
      x_1 \quad &+ \quad &3 x_2 \quad &= \quad 28
    \end{alignat}
\end{enumerate}

\section{Calculus}
\begin{enumerate}[\thesection .1]
\item Find the derivative of
  \begin{equation*}
    f(x) = 25 x^{22} + 12 x^8 + 17 + \frac{9}{x^5}
  \end{equation*}

\item For $f(x, y, t) = x^2 t^3 + 4 x y t^2 + 2 y^2 t + y^3$
  \begin{enumerate}
  \item Find $\frac{\partial f}{\partial x}$
  \item Find $\frac{\partial f}{\partial y}$
  \item Find $\frac{\partial f}{\partial t}$
  \item Find $\frac{\partial ^2 f}{\partial x \partial t}$
  \end{enumerate}

\item Differentiate
  \begin{equation*}
    f(x) = x e^{-x^3}
  \end{equation*}

\item Differentiate
  \begin{equation*}
    f(\theta) = cos^2(\theta) sin(2 \theta)
  \end{equation*}

\item Evaluate $\int x^2\,{dx}$

\item Evaluate
  \begin{equation*}
    \int_{0}^{2 \pi} 2\,cos(\theta)\,sin(\theta)\,d\theta
  \end{equation*}

\end{enumerate}

\section{Differential Equations}
\begin{enumerate}[\thesection .1]
\item Solve the following
  \begin{equation*}
    \frac{du}{dx} = c u + x^2
  \end{equation*}

\item Solve the following equation and identify it's use
  \begin{equation*}
    \frac{d^2u}{dx^2} + {\omega}^2 u = 0
  \end{equation*}

\item Solve the following equation and identify it's use. What can you say about the constant $k$ in its effect on the solution?
  \begin{equation*}
    \frac{dA}{dt} = - k A
  \end{equation*}

\item Solve the following
   \begin{equation*}
    m \frac{dv}{dt} = F
  \end{equation*}

\item Solve the following
  \begin{equation*}
    \frac{dy}{dx} = x^2 (1 + y),\quad y(0) = 3
  \end{equation*}

\item Find the general solution to
  \begin{equation*}
    \frac{du}{dt} = \alpha (1 - u) - \beta u
  \end{equation*}

\item Solve the following
  \begin{equation*}
    \frac{dy}{dx} + P(x) y = Q(x) y^n
  \end{equation*}
\end{enumerate}

%\section{Graph Theory and Complexity}
%\begin{enumerate}[\thesection .1]
%\item How many vertices and nodes are in the following graph?
%  \\ \\
%  \begin{tikzpicture}[node distance={15mm}, thick, main/.style = {draw, circle}]
%    \node[main] (1) {$x_1$};
%    \node[main] (2) [above right of=1] {$x_2$};
%    \node[main] (3) [below right of=1] {$x_3$};
%    \node[main] (4) [above right of=3] {$x_4$};
%    \node[main] (5) [above right of=4] {$x_5$};
%    \node[main] (6) [below right of=4] {$x_6$};
%
%    \draw[->] (1) -- (2);
%    \draw[->] (1) -- (3);
%    \draw[->] (1) to [out=135, in=90, looseness=1.5] (5);
%    \draw[->] (1) to [out=180, in=270, looseness=5] (1);
%    \draw[->] (2) -- (4);
%    \draw[->] (3) -- (4);
%    \draw[->] (5) -- (4);
%    \draw[->] (5) to [out=315, in=315, looseness=2.5] (3);
%    \draw[->] (6) -- (4);
%  \end{tikzpicture}
%
%\item Is the graph from the previous exercise \textbf{\textit{directed}} or \textbf{\textit{undirected}}? Justify your answer.
%
%\item What can you say about the reachability of node 6?
%
%\item What is the result of reaching node 4?
%
%\item In the following graph,
%  \\ \\
%  \begin{tikzpicture}[auto, node distance={25mm}, thick, main/.style = {draw, circle}]
%    \node[main] (1) {$x_1$};
%    \node[main] (2) [above right of=1] {$x_2$};
%    \node[main] (3) [below right of=1] {$x_3$};
%    \node[main] (4) [below right of=2] {$x_4$};
%    \node[main] (5) [above right of=4] {$x_5$};
%    \node[main] (6) [below right of=4] {$x_6$};
%    \node[main] (7) [above right of=6] {$x_7$};
%
%    \draw[->] (1) to node {+1} (2);
%    \draw[->] (1) to node {+2} (3);
%    \draw[->] (3) to node {+2} (4);
%    \draw[->] (2) to node {+5} (5);
%    \draw[->] (5) to node {+1} (4);
%    \draw[->] (4) to node {+3} (6);
%    \draw[->] (3) to node {+3} (6);
%    \draw[->] (5) to node {+2} (7);
%    \draw[->] (6) to node {+1} (7);
%    
%  \end{tikzpicture}
%  \\ \\
%  \begin{enumerate}
%  \item What is the cost going from node $x_1$ to node $x_7$ via node $x_6$?
%  \item What is the cost of the path $x_1$, $x_2$, $x_5$, $x_7$?
%  \item Which path has the lowest cost of going from node $x_1$ to node $x_7$?
%  \end{enumerate}
%\item If an algorithm for traversing a graph is described as having \textbf{\textit{time complexity}} as $O(n)$, what does this mean?
%
%\item If an algorithm for traversing a graph is described as having \textbf{\textit{space complexity}} as $O(2^n)$, what does this mean?
%
%\item Briefly describe what is meant when a problem is described as being \textbf{\textit{NP-complete}}.
%
%
%\end{enumerate}


\section{Probability and Statistics}
\begin{enumerate}[\thesection .1]
\item A cloth bag has 3 green marbles, 2 blue marbles, 4 yellow marbles, 6 red marbles, and 5 purple marbles.
  \begin{enumerate}
  \item If you draw 1 marble 10 times (replacing the marble each time), how many times would you expect to get a marble that is not green or blue?
  \item If you draw 1 marble 10 times (without replacing the marble each time), how many times would you expect to get a marble that is not red?
  \end{enumerate}

\item
  Describe the difference between a deterministic and a probabilistic statistical model. Under what conditions would you choose a deterministic model over a probabilistic model?

\item Following is a statement of Bayes' Theorem \\ \\
  \begin{equation*}
    P(H \, \vert \, E) = \frac {P(E \, \vert \, H) \cdot P(H)}{P(E)}
  \end{equation*} \\ \\
  \begin{enumerate}
  \item Give a description of each of the terms, including $E$ and $H$.
  \item What is the value of using a Bayesian model versus using other statistical models?
  \end{enumerate}

\item What is the Markov property?

%\item
\end{enumerate}

%\section{Optimization}
%\begin{enumerate}[\thesection .1]
%  \item 
%\end{enumerate}

\end{document}
