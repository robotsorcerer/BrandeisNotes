\chapter{How to Optimally rotate two vectors.}  
 \label{chap:back}
 
 We are chiefly concerned with \textit{rigid bodies} and occasionally \textit{semi-rigid} or \textit{soft} bodies connected together by \textit{joints}.   In general, we take robots to be \textit{mechanisms} that are made up of \textit{links} connected to one another by \textit{joints}.  Typically, the joints connect two or more links and are formed by simple contact with adjacent bodies.  Sometimes, the joints may be flexible -- whether by belt, band, spring or some kind of elastic component such as bellows, diaphragms, tendons~\cite{Bern17ACM}, fiber-reinfoced elastomers~\cite{BishopFREE2012}, resilient pads, strip, or bush. The assembly formed after the various connections between links and joints are called  a \textit{kinematic chain}. 