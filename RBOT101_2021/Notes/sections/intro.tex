\chapter{Preamble}
\label{chap:intro}

Consider this the roadmap for this course.  Please read through the syllabus posted on Moodle2 carefully and feel free to share any questions that you may have.  Please print a copy of the Syllabus for reference. Some relevant parts of the Syllabus are repeated here but the Moodles reference should serve as your guide throughout the ten weeks of this course.

\section{Course Description}
%
This course focuses on the algorithmic and mathematical concepts  with respect classical and recent methods for solving real-world problems in robotics. While some students may have encountered some of the concepts we will be treating in past courses or avenues of study, we will provide the breadth and depth necessary for equipping students to be world-class roboticists. The topics covered by this course shall include the configuration space, rigid bodies, semi-rigid soft bodies, as well as their motions in $\mathbb{R}^n$,  wrenches, homogeneous transformations, optimal algorithms for rigid body rotations, linear systems theory, probability theory, the Kalman filter. The course will begin and end with a self-assessment to allow students to gauge their strengths and weaknesses in these topics. References for further, in-depth study in each topic are provided at the end of this course.

%This course focuses on the algorithmic and mathematical concepts  with respect to robot planning, manipulation and control. Topics covered include kinematics and dynamics, as well as path planning and deep reinforcement learning algorithms. Simulations and experiments on hardware testbeds (listed in the syllabus) will be performed to test the related algorithms.

\section{Course Outcomes}
After taking this course, each student will be able to

\begin{itemize}
\item Develop mathematical tools for solving fundamental kinematic problems in robot operation;
%
\item  Formulate optimal state estimation tools for solving real-time smoothing and filtering operations in robotics;
%
\item Integrate state estimation with rigid and semi-rigid soft bodies to solve real-world automation problems; and
%
4.	Use open-source Python, and C++ tools to solve classical and emerging problems in robotics in our day.
\end{itemize}

\section{Prerequisites}

An undergraduate-level understanding of linear algebra, analytical mechanics, Python and C++ programming.

\section{Recommended Texts}
\begin{itemize}
	\item  	Main Texts
	\begin{itemize}
		\item Simon, Dan. (2007). Optimal state estimation: Kalman, $H-\infty$, and nonlinear approaches. Choice Reviews Online, Vol. 44, pp. 44-3334-44–3334. https://doi.org/10.5860/choice.44-3334
		%
		\item Murray, R. M., Li, Z., and Sastry, S. S. (1994). A Mathematical Introduction to Robotic Manipulation. Book (Vol. 29). Free PDF preprint downloadable from, \href{https://www.cds.caltech.edu/~murray/books/MLS/pdf/mls94-complete.pdf }{Murray's website}.
		%
		%\item Modern Robotics: Mechanics, Planning, and Control. Free PDF preprint downloadable from the \href{ http://hades.mech.northwestern.edu/images/7/7f/MR.pdf}{author's website}
		%
		\item Theory of Screws: A Study in the Dynamics of a Rigid Body by Robert Stawell Ball, Dublin: Hodges, Foster, and Co., Grafton-Street. a. Textbooks:
	\end{itemize} 
	%
	\item Secondary Text
	%
	\begin{itemize}
		\item Modern Robotics: Mechanics, Planning, and Control. Free PDF preprint downloadable from \href{ http://hades.mech.northwestern.edu/images/7/7f/MR.pdf}{Author's Northwestern University Website}.		
	\end{itemize} 
    %
    \item 
    Auxiliary Text: 
    %
    \begin{itemize}
    	\item Theory of Screws: A Study in the Dynamics of a Rigid Body by Robert Stawell Ball, Dublin: Hodges, Foster, and Co., Grafton-Street (Should be downloadable via Interlibrary Loan).
    \end{itemize}
\end{itemize}

\section{Recommended Journals}
	%
	\begin{itemize}
		\item 
		\href{ https://ieeexplore.ieee.org/xpl/RecentIssue.jsp?punumber=8860}{IEEE Transactions on Robotics}.
		%
		\item 
		\href{https://journals.sagepub.com/home/ijr}{The International Journal of Robotics Research}.
		%
		\item 
		\href{https://www.ieee-ras.org/conferences-workshops/fully-sponsored/icra}{The IEEE International Conference on Robotics and Automation (ICRA)}.
		%
		\item \href{https://www.ieee-ras.org/conferences-workshops/financially-co-sponsored/iros}{IEEE/Robotics Society of Japan International Conference on Intelligent Robots and Systems (IROS)}.
		%
		\item \href{https://www.journals.elsevier.com/robotics-and-autonomous-systems}{Robotics and Autonomous Systems, An Elsevier Journal}.
	\end{itemize}

\section{Required Software}
	
	\begin{itemize}
	%
	\item A working knowledge of python and the anaconda environment.
	%
	\item ROS 1.x Installation Instructions:  \href{https://www.ros.org/}{ros 1.x website}.
	%
	\item ROS 2 installation \href{https://index.ros.org/doc/ros2/Installation/Crystal/Linux-Install-Binary/}{ros 2.0 website}.
	\end{itemize}

\section{Online Course Content}
%
This course will be conducted completely online using Brandeis’ LATTE \href{http://moodle2.brandeis.edu}{site}. The site contains the course syllabus, assignments, our discussion forums, links/resources to course-related professional organizations and sites, and weekly checklists, objectives, outcomes, topic notes, self-tests, and discussion questions.  Access information is emailed to enrolled participants before the start of the course.   To begin participating in the course, review the ``Welcoming Message" and the ``Week 1 Checklist."

%\section{Assessments and Labs}
%
%Please read the syllabus posted on the RBOT 101 website thoroughly.

\section{Errata}

If in the course of using these notes, you find sentence errors, errata or mistakes in equations, please annotate them and upload it to the discussion forum. Points will awarded, at the discretion of the instructor, for such help.