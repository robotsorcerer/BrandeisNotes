\noindent 

\begin{homework}
	Show that $\bm{x}^i$ are mutually orthogonal and normalized \ie orthonormal for the following $N$-dimensional Euclidean basis coordinate vectors
	\begin{align}
		\bm{x}^1 = \begin{bmatrix}
			1 \\ 0 \\ \vdots \\ 0
		\end{bmatrix}
		%
		\qquad
		%
		\bm{x}^2 = \begin{bmatrix}
			0 \\ 1 \\ \vdots \\ 0
		\end{bmatrix}
		%
		\qquad
		%
		\bm{x}^N = \begin{bmatrix}
			0 \\ 0 \\ \vdots \\ 1
		\end{bmatrix}
	\end{align}
\end{homework}

\begin{solution}
	To prove orthogonality, we must have %
	
\end{solution}