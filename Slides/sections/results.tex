
\section{Results}

\begin{frame}
	\frametitle{Results}
	\begin{itemize}
		%\item We present results on 3D prostate cases
		%
		\item Start by randomly adding five beam blocks to the state queue
		%
		\item Input planes are passed through the tower residual network, from which probability distributions and a value are predicted
		%
		\item Add a random walk sequence to the generated pure strategy 
		%
		\item Construct tree with this mixed strategy %(at subsequent iterations, a previous iteration of the neural network is used to generate the mixed strategy used to run the tree search)
		%
		\item As new beam angle combinations are found according to the MCTS Algorithm, the FIFO queue is updated
	\end{itemize}
\end{frame}


\newcommand{\putdose}[2]{\includegraphics[width=#2\columnwidth, height=.28\columnwidth]{../../BOO/figures/dvh_dose/#1}}
\newcommand{\dosewidth}{.35}

\begin{table}[tb!]
	\footnotesize{\textbf{Dose washes for select patients during training of the self-play network}}
	\centering
	\begin{tabular}{@{}c@{}c@{}c@{}}
		\toprule
		\midrule
		& \textbf{Training Regime} & \\
		%
		\putdose{case_007/dose.png}{\dosewidth} & \putdose{case_011/dose.png}{\dosewidth} & \putdose{case_017/dose.png}{\dosewidth} 
		%
		\\
		\putdose{case_022/dose.png}{\dosewidth} & \putdose{case_025/dose.png}{\dosewidth}  & \putdose{case_065/dose.png}{\dosewidth}  \\
		\midrule
		\bottomrule
	\end{tabular}
	\label{tbl:dose}
	%
\end{table}


\begin{table}[tb!]
	\footnotesize{\textbf{Dose washes for select patients during testing of self-play network}}
	\begin{tabular}{@{}c@{}c@{}c@{}}
		\toprule
		\midrule
		& \textbf{Inference Regime} & \\
		\putdose{case_062/dose.png}{\dosewidth}  & \putdose{case_066/dose.png}{\dosewidth} & \putdose{case_073/dose.png}{\dosewidth} 
		%
		\\
		\putdose{case_075/dose.png}{\dosewidth}  & \putdose{case_077/dose.png}{\dosewidth} & \putdose{case_078/dose.png}{\dosewidth}	\\	
		\midrule
		\bottomrule
	\end{tabular}
	\label{tbl:dose_test}
\end{table}

\begin{frame}
	\frametitle{Results}
	\begin{itemize}
		\item The policy selects fairly equidistant beams
		%
		\item Yields wash plots that provide good dosimetric concentration on the tumor
		%
		\item Gives sharp gradients at transition between tumors and OARs
		%
		\item Largely avoids strong dose to OARs
	\end{itemize}
\end{frame}

\begin{frame}
\frametitle{Results}
\begin{itemize}
	\item Finding the good beam angle candidates is orders of magnitude faster than the current approaches
	%
	\vspace{0.3cm}
	\begin{itemize}
		\item Mostly due to neural network oracle
		%	%
		\vspace{0.3cm}
		\item At inference, we pick the last checkpoint  during training and use it to find feasible beam angles
		%	%
		\vspace{0.3cm}
		\item Beam angles prediction now takes between 2-3 minutes before we settle on a good candidate beam angle set. 
	\end{itemize}
	%
\end{itemize}
\end{frame}


\begin{frame}
\frametitle{Transition Slide}
\centering This page is left blank intentionally.
\end{frame}