\definecolor{light-blue}{rgb}{0.30,0.35,1}
\definecolor{light-green}{rgb}{0.20,0.49,.85}
\definecolor{purple}{rgb}{0.70,0.69,.2}

\newcommand{\lb}[1]{\textcolor{light-blue}{#1}}
\newcommand{\bl}[1]{\textcolor{blue}{#1}}

\newcommand{\maybe}[1]{\textcolor{gray}{\textbf{MAYBE: }{#1}}}
\newcommand{\inspect}[1]{\textcolor{cyan}{\textbf{CHECK THIS: }{#1}}}
\newcommand{\more}[1]{\textcolor{red}{\textbf{MORE: }{#1}}}
\renewcommand{\figureautorefname}{Fig.}
\renewcommand{\sectionautorefname}{$\S$}
\renewcommand{\equationautorefname}{eq.}
\renewcommand{\subsectionautorefname}{$\S$}

% FYA
\newcommand{\cmt}[1]{{\footnotesize\textcolor{red}{#1}}}%{#2}
%\newcommand{\note}[1]{\cmt{Note: #1}}
\newcommand{\todo}[1]{\textcolor{cyan}{TO-DO: #1}}
\newcommand{\lekan}[1]{\cmt{\textbf{LO}: #1}}
\newcommand{\nick}[1]{\cmt{\textbf{NG}: #1}}
\newcommand{\mike}[1]{\cmt{\textbf{Mike}: #1}}
\newcommand{\steve}[1]{\cmt{\textbf{SJ}: #1}}
\newcommand{\review}[1]{\noindent\textcolor{red}{$\rightarrow$ #1}}
\newcommand{\response}[1]{\noindent{#1}}
\newcommand{\stopped}[1]{\color{red}STOPPED HERE #1\hrulefill}

%Text commands
\newcounter{mnote}
\newcommand{\marginote}[1]{\addtocounter{mnote}{1}\marginpar{\themnote. \scriptsize #1}}
\setcounter{mnote}{0}
% \newcommand{\comment}[1]{}
\newcommand{\ie}{$i.e.$\ }
\newcommand{\eg}{e.g.\ }
\newcommand{\cf}{c.f.\ }
\newcommand{\yes}{\checkmark}
\newcommand{\no}{\ding{55}}

%Reference commands
\newcommand{\flabel}[1]{\label{fig:#1}}
\newcommand{\seclabel}[1]{\label{sec:#1}}
\newcommand{\tlabel}[1]{\label{tab:#1}}
\newcommand{\elabel}[1]{\label{eq:#1}}
\newcommand{\alabel}[1]{\label{alg:#1}}
\newcommand{\fref}[1]{\cref{fig:#1}}
\newcommand{\sref}[1]{\cref{sec:#1}}
\newcommand{\tref}[1]{\cref{tab:#1}}
\newcommand{\eref}[1]{\cref{eq:#1}}
\newcommand{\aref}[1]{\cref{alg:#1}}

% \newcommand{\wtsa}{\theta_{\infn}}
% \newcommand{\wtsb}{\theta_{\outfn}}
\newcommand{\bull}[1]{$\bullet$ #1}
\newcommand{\argmax}{\text{argmax}}
\newcommand{\argmin}{\text{argmin}}
\newcommand{\mc}[1]{\mathcal{#1}}
\newcommand{\bb}[1]{\mathbb{#1}}
\newcommand{\deq}{\mathrel{\stackrel{\text{\tiny{def}}}{=}}}
\newcommand{\opequals}[1]{\ #1\hspace{-3pt}=\hspace{1pt}}
\newcommand{\plusequals}{\opequals{+}}
\newcommand{\minusequals}{\opequals{-}}
\newcommand{\timesequals}{\opequals{*}}
% \newcommand{\minusequals}{\ -\hspace{-3.2pt}=\hspace{1pt}}
% \newcommand{\timesequals}{\ *\hspace{-3.2pt}=\hspace{1pt}}
\newcommand{\front}[1]{beg(#1)}
\newcommand{\back}[1]{end(#1)}
\newcommand{\stddev}{\sigma}
\newcommand{\variance}{\stddev^2}
\newcommand{\ifelse}[3]{\begin{cases}#1 \text{ if } #2\\#3 \text{ otherwise}\end{cases}}


%Figure commands
\newcommand{\figdir}{figures/}
\newcommand{\capt}[2]{\caption[#1]{#1#2}}
\newcommand{\qcapt}[1]{\capt{#1}{}}
\newcommand{\figheadingnospace}[1]{\center{\textbf{#1}}}
\newcommand{\figheading}[1]{\figheadingnospace{#1}\vspace{3mm}}
\newcommand{\fig}[5]
{
\begin{figure}
\begin{center}
\includegraphics[width=#3\columnwidth]{figures/#1}
\end{center}
\capt{#4}{#5}
\flabel{#2}
\end{figure}
}
\newcommand{\figtbph}[5]
{
\begin{figure}[tbph]
\begin{center}
\includegraphics[width=#3\columnwidth]{figures/#1}
\end{center}
\capt{#4}{#5}
\flabel{#2}
\end{figure}
}
\newcommand{\figt}[5]
{
\begin{figure}[tb]
\begin{center}
\includegraphics[width=#3\columnwidth]{figures/#1}
\end{center}
\capt{#4}{#5}
\label{#2}
\end{figure}
}
\newcommand{\figstar}[5]
{
\begin{figure*}
\begin{center}
\includegraphics[width=#3\textwidth]{figures/#1}
\end{center}
\capt{#4}{#5}
\flabel{#2}
\end{figure*}
}
\def\tidx{t}
%\def\value{V}
% from https://www.cs.jhu.edu/~jason/advice/write-the-paper-first.html
\newcommand{\Note}[1]{}
\renewcommand{\Note}[1]{\hl{[#1]}}  % comment out this definition to suppress all Notes
\newcommand{\NoteSigned}[3]{{\sethlcolor{#2}\Note{#1: #3}}}
\newcommand{\NoteLO}[1]{\NoteSigned{LO}{YellowGreen}{#1}}  % use for notes from Lekan
\newcommand{\NoteDN}[1]{\NoteSigned{DN}{LightBlue}{#1}}    % use for notes from Dan
%
\newcommand{\bolditalic}[1]{\textbf{#1}}
\theoremstyle{definition}
%
\newcommand{\emptybox}[2][\textwidth]
{%
	\begingroup
	\setlength{\fboxsep}{-\fboxrule}%
	\noindent\framebox[#1]{\rule{0pt}{#2}}%
	\endgroup
}


\def\dof{\text{DOF }}
\def\dofs{\text{DOFs }}
\def\rot{{R}}
\def\rthree{\bb{R}^3}
\def\ren{\bb{R}^n}
\def\disp{{d}}
\def\homo{g}
\def\homost{g_{st}}
\def\skew{S}
\def\jacob{J}
\def\twistcoord{\xi}
\def\twistmap{\hat{\xi}}
\def\bodyform{\mathcal{B}}
\def\identity{I}
\newcommand{\fwdmap}[2]{{e^{\twistmap_{#1} #2_#1} }}
\newcommand{\fwdmaptheta}[1]{{e^{\twistmap_{#1} \theta_#1} }}
%\def\fwdmap[1]{e^{\twistmap_{#1} } }
