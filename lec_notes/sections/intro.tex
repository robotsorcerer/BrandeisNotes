\chapter{Preamble}
\label{chap:intro}

Consider this your roadmap for the course.  Please read through the syllabus posted on moodle carefully and feel free to share any questions that you may have.  Please print a copy of the syllabus for reference. Some relevant parts of the syllabus are repeated here but the moodle reference should serve as your guide throughout the ten weeks of this course.

\section{Course Description}
This course focuses on the algorithmic and mathematical concepts  with respect to robot planning, manipulation and control. Topics covered include kinematics and dynamics, as well as path planning and deep reinforcement learning algorithms. Simulations and experiments on hardware testbeds (listed in the syllabus) will be performed to test the related algorithms.

\section{Course Outcomes}
After taking this course, each student will be able to

\begin{itemize}
\item Develop planning and manipulation schemes to drive robot operation

\item Integrate perception algorithms into manipulation and planning systems

\item Determine the kinematic description of a robot's motion or locomotion
\end{itemize}

\section{Prerequisites}

RBOT 210 or an advanced knowledge of ROS; undergraduate-level experience or equivalent with object oriented programming; strong programming knowledge in Python and C++ is required; and RBOT 205 if mathematical foundational skills of admissions criteria are needed.

\section{Recommended Texts}
\begin{itemize}
	\item  	Main Texts
	\begin{itemize}
		\item Murray, R. M., Li, Z., and Sastry, S. S. (1994). A Mathematical Introduction to Robotic Manipulation. Book (Vol. 29). Free PDF preprint downloadable from, \href{https://www.cds.caltech.edu/~murray/books/MLS/pdf/mls94-complete.pdf }{Murray's website}.
		%
		\item 	Spong, M. W., Hutchinson, S., and Vidyasagar, M. (2012). Robot Modeling and Control. Students can buy from this \href{https://www.amazon.com/Robot-Modeling-Control-Mark-Spong/dp/0471649902}{Amazon Link}.
	\end{itemize} 
	%
	\item Secondary Text
	%
	\begin{itemize}
		\item Modern Robotics: Mechanics, Planning, and Control. Free PDF preprint downloadable from \href{ http://hades.mech.northwestern.edu/images/7/7f/MR.pdf}{Author's Northwestern Website}.		
	\end{itemize} 
    %
    \item 
    Auxiliary Text: 
    %
    \begin{itemize}
    	\item Theory of Screws: A Study in the Dynamics of a Rigid Body by Robert Stawell Ball, Dublin: Hodges, Foster, and Co., Grafton-Street (Should be downloadable via Interlibrary Loan).
    \end{itemize}
\end{itemize}

\section{Recommended Journals}
	%
	\begin{itemize}
		\item 
		\href{ https://ieeexplore.ieee.org/xpl/RecentIssue.jsp?punumber=8860}{IEEE Transactions on Robotics}.
		%
		\item 
		\href{https://journals.sagepub.com/home/ijr}{The International Journal of Robotics Research}.
		%
		\item 
		\href{https://www.ieee-ras.org/conferences-workshops/fully-sponsored/icra}{The IEEE International Conference on Robotics and Automation}.
		%
		\item \href{https://www.ieee-ras.org/conferences-workshops/financially-co-sponsored/iros}{IEEE/Robotics Society of Japan International Conference on Intelligent Robots and Systems (IROS)}.
		%
		\item \href{https://www.journals.elsevier.com/robotics-and-autonomous-systems}{Robotics and Autonomous Systems, An Elsevier Journal}.
	\end{itemize}

\section{Required Software}
	
	\begin{itemize}
	%
	\item A working knowledge of python and the anaconda environment.
	%
	\item \href{https://www.ros.org/}{The Robot Operating System}.
	%
	\item ROS from Conda installation \href{ https://medium.com/@wolfv/ros-on-conda-forge-dca6827ac4b6}{instructions}.
	\end{itemize}

\section{Online Course Content}
%
This course will be conducted completely online using Brandeis’ LATTE \href{http://moodle2.brandeis.edu}{site}. The site contains the course syllabus, assignments, our discussion forums, links/resources to course-related professional organizations and sites, and weekly checklists, objectives, outcomes, topic notes, self-tests, and discussion questions.  Access information is emailed to enrolled participants before the start of the course.   To begin participating in the course, review the ``Welcoming Message" and the ``Week 1 Checklist."

\section{Assessments and Labs}

Please read the syllabus posted on the RBOT 250 website thoroughly.

\section{Errata}

If in the course of using these notes, you find sentence errors, errata or mistakes in equations, please annotate them and upload it to the discussion forum. Points will awarded, at on the discretion of the instructor, for such help.